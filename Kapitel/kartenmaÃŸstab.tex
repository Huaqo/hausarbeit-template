\section{Kartenmaßstab}


Kohlstock zeigt Formel \ref{fig:equa1} und sagt: “Ist der Maßstab einer Karte nicht bekannt oder soll er an einer bestimmten Kartenstelle überprüft werden, so kann er durch einen Streckenvergleich bestimmt werden. Hierbei erhält man die Maßstabszahl m [...]". \cite{kohlstock_kartographie_2014}

\begin{equation}
\centering
m=\frac{s}{s_{k}}=\frac{\mbox {Naturstrecke}}{\mbox {Kartenstrecke}}=\frac{s^{*} \cdot m^{*}}{s_{k}} 
\label{fig:equa1}
\end{equation} % \mbox statt \text

Die Strecke zwischen Königstein (B) und Lilienstein (A) beträgt in Wirklichkeit 2100 m. Auf der abgebildeten Karte beträgt die Strecke 8,4 cm. Daraus folgt:

\begin{equation}
\centering
m=\frac{\mbox {210000 cm}}{\mbox {8,4 cm}}={\mbox{25000}}
\label{fig:equa2}
\end{equation}

Der berechnete Kartenmaßstab aus Formel \ref{fig:equa2} ist 1:25000. Eine Bestätigung für den Originalmaßstab gibt \cite{kohlstock_kartographie_2014}
, er bezeichnet diesen Maßstab als Grundkarte (Primärkarte), so “[...] bezeichnet man das grundlegende topographische Kartenwerk eines Landes, aus welchem die Karten kleineren Maßstabs abgeleitet werden. Sie präsentieren die Landesaufnahme und haben i.d.R. einen Maßstab M>1:25.000". Es ist auch typisch für topographische Karten einen Maßstab von 1:25.000 zu haben \cite{jensch_erde_1975}. Zuletzt wird der Originalmaßstab der Karte in dessen Legende mit dem Maßstab 1:25000 bestätigt  \cite{landesvermessungsamt_sachsen_topographische_2009}.